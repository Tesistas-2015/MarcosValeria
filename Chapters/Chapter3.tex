% Chapter Template

\chapter{M\'etodos} % Main chapter title

En este capitulo describiremos el algoritmo utilizado para la realizacion del presente trabajo final.\\

En la secci\'on 3.1 realizaremos una descripci\'on general, la misma contiene informaci\'on acerca de  los m\'etodos de segmentaci\'on propuestos para las im\'agenes de fondo de ojo utilizadas en este proyecto, el concepto de preprocesamiento de im\'agenes, la extracci\'on de indicadores y finalmente el entrenamiento y clasificaci\'on de las im\'agenes basado en indicadores.\\

En la secci\'n 3.2 nos enfocaremos en el preprocesamiento de las im\'agenes. Se detallan las caracteristicas iniciales de las im\'agenes propuestas, las t\'ecnicas utilizadas para la extracci\'on de ruido y fondo, as\'i como tambi\'en la tecnica utilizada para mejorar el contraste. Finalmente se explican los pipelines propuestos y analizados para el preprocesamiento correspondiente de las im\'agenes de fondo de ojo.\\

En la secci\'on 3.3...\\

En la secci\'on 3.4...\\
\label{Chapter3} % Change X to a consecutive number; for referencing this chapter elsewhere, use \ref{ChapterX}

%----------------------------------------------------------------------------------------
%	SECTION 1
%----------------------------------------------------------------------------------------

\section{Descripci\'on general}

Una categorizaci\'on com\'un de los algoritmos para segmentaci\'on de estructuras vasculares en im\'agenes m\'edicas incluyen t\'ecnicas, tales como enforques basados en regiones y basados en bordes, t\'ecnicas de reconocimiento de patrones, enfoques basados en modelos, enfoques basados en el seguimiento y enfoques basados en redes neuronales. Los algoritmos para la segmentaci\'on de vasos sanguineos se pueden dividir en 6 categorias principales:
\begin{itemize}
	\item T\'ecnica de reconocimiento de patrones.
	\item Filtrado adaptado.
	\item Seguimiento de vasos.
	\item Morfolog\'ia matem\'atica.
	\item Enfoques multiescala.
	\item Enfoques basados en modelos.
	\item Aproximaciones paralelas basadas en hardware.
\end{itemize}
	
La t\'ecnica de reconocimiento de patrones se divide en dos subcategorias, enfoques supervisados y enfoques no supervisado.

El m\'etodo de segmentaci\'on propuesto en el presente trabajo final es un m\'etodo supervisado. En este m\'etodo, la regla para la extracci\'on de los vasos sanguineos es aprendido sobre la base de un conjunto de formaci\'on de im\'agenes de referencia procesados y segmentados manualmente por un oftalm\'ologo. En un procedimiento supervisado, los criterios de clasificaci\'on son determinados sobre la base de caracter\'isticas dadas. Por lo tanto, el requisito previo es la disponibilidad de los datos reales ya clasificados. Como m\'etodos supervisados est\'an dise\~nados en base a los datos pre-clasificados, su rendimiento suele ser mejor que el de los m\'etodos no supervisados y puede producir muy buenos resultados para las im\'agenes retinianas saludables. \cite{fraz2012blood} \cite{akita1982computer} \cite{hoover2000locating} \\

FIGURA\\

El procesamiento digital de im\'agenes es el conjunto de t\'ecnicas que se aplican a las im\'agenes digitales con el objetivo de mejorar la calidad o facilitar la b\'usqueda de informaci\'on.\\

Durante el preprocesamiento de im\'agenes se aplican un conjunto de filtros, cuyo objetivo fundamental es obtener, a partir de una imagen inicial, otra final cuyo resultado sea mas adecuado para una aplicaci\'on espec\'ifica.

En este trabajo, el preprocesamiento tiene por objetivo:
\begin{itemize}
 \item Suavizar la imagen: reducir la cantidad de variaciones de intensidad entre p\'ixeles vecinos.
 \item Eliminar ruido: eliminar aquellos p\'ixeles cuyo nivel de intensidad es muy diferente al de sus vecinos y cuyo origen puede estar tanto en el proceso de adquisici\'on de la imagen como en el de transmisi\'on.
 \item Realzar el contraste entre los vasos sanguineos y las demas estructuras anat\'omicas que componen la retina.
\end{itemize}








%-----------------------------------
%	SECTION 2
%-----------------------------------
\section{Preprocesamiento}

La fotografía retinal requiere el uso de un complejo sistema óptico, llamado camara de fondo. Este es  un microscopio especializado  de baja potencia con una cámara adjunta, capaz de iluminar y formar imágenes de la retina simultáneamente. Está diseñado para tomar una imagen de la superficie interior del ojo, que incluye la retina, el disco óptico, mácula y polo posterior. La cámara de fondo normalmente opera en tres modos. En la fotografía a color de la retina se examina todo bajo la iluminación de luz blanca. En la fotografía libre de rojo, los vasos y otras estructuras se mejoran en contraste y la luz de la imagen se filtra para eliminar los colores rojos. Las angiografías con fluoresceína se adquieren mediante el método del colorante de rastreo. La fluoresceína de sodio y verde de indocianina se inyecta en la sangre, y luego el angiograma se obtiene fotografiando la fluorescencia emitida después de la iluminación de la retina con luz azul a una longitud de onda de 490 nanómetros.\\

La vascularización retiniana se compone de arterias y venas que aparecen como rasgos alargados, con sus afluentes visibles dentro de la imagen retiniana. Hay una amplia gama de anchos de los vasos que van desde un píxel a veinte píxeles, dependiendo de la anchura de la embarcación y la resolución de la imagen. Otras estructuras que aparecen en las imágenes de fondo de ojo incluyen el límite retina, el disco óptico, y las patologías en forma de manchas algodonosas, lesiones brillantes y oscuras y exudados como se muestra en (\ref{fig:Fondo_de_ojo} (b-d)).\\

El recipiente de perfiles de intensidad de la sección transversal se aproximan a una forma gaussiana o una mezcla de gaussianas en el caso en que un reflejo central de vasos se encuentre presente. La orientación y la escala de grises de un vaso no cambia bruscamente; que son localmente lineales y cambian gradualmente en intensidad a lo largo de su longitud. Los vasos se pueden esperar para ser conectado y, en la retina, forman un árbol binario como la estructura. Sin embargo, la forma, tamaño y nivel de gris local de los vasos sanguíneos pueden variar enormemente y algunas de las características de fondo pueden tener atributos similares a los vasos como se ilustra en (\ref{fig:Fondo_de_ojo} (a y d)).
El cruce de vasos y de ramificación pueden complicar aún más el modelo de perfil. Al igual que con el procesamiento de la mayoría de las imágenes médicas, la señal de ruido, la deriva en intensidad de la imagen y la falta de contraste de imagen plantean retos importantes para la extracción de vasos sanguíneos.
Los vasos de la retina también muestran una evidencia de una fuerte reflexión a lo largo de su línea central conocida como un reflejo vaso central, como es evidente en (\ref{fig:Fondo_de_ojo}  (a)), que es más evidente en las arterias que las venas, es más fuerte en las imágenes tomadas en longitudes de onda más largas, y se encuentran típicamente en las imágenes de la retina de los pacientes más jóvenes.\cite{fraz2012blood}\\


\begin{figure}[H]
	{
	\centering
	\includegraphics[width=1\textwidth]{Figures/image_ojo}
	\caption[Imagenes de Fondo de ojo]{Morfolog\'ia de im\'agenes de la retina: (a)Reflejo en el vaso central y fondo desigual, (b) Manchas algodonosas, (c)Exudados, (d) Estructura anat\'omica de la retina.}
	\label{fig:Fondo_de_ojo}
	}
\end{figure}	


Los m\'etodos de segmentaci\'on de los vasos de la retina son evaluados en tres bases de datos, DRIVE, HRF y ARIA.

DRIVE es una base de datos a disposici\'on del p\'ublico, que consta de un total de 40 im\'agenes a color con lesiones. Las fotograf\'ias se obtuvieron a partir de un programa de cribado de la retinopat\'ia diab\'etica en los Pa\'ises Bajos.
La poblaci\'on de selecci\'on consisti\'o en 453 sujetos de entre 31 y 86 a\~nos de edad. Cada imagen se ha comprimido JPEG , que es pr\'actica com\'un en los programas de cribado.\\
De las 40 im\'agenes en la base de datos, 7 contienen la patolog\'ia, es decir, los exudados, hemorragias y cambios del epitelio pigmentario. Ver (\ref{fig:Drive_images_retinal}) para un ejemplo de una imagen patol\'ogica y  una normal. Las im\'agenes fueron adquiridas mediante una camara no midri\'atica.\\

\begin{figure}[H]
	{
	\centering
	\includegraphics[width=1\textwidth]{Figures/Drive_images_retinal}
	\caption[Drive]{Im\'agen de la retina - DRIVE: (a) Retina saludable, (b) Retina con muestras patol\'ogicas.}
	\label{fig:Drive_images_retinal}
	}
\end{figure}	



En primer lugar, se determinaron los algoritmos a utilizar para mejorar el preprocesamiento de las im\'agenes, es decir, disminuir los efectos secundarios que se generan en la captura de im\'agenes de fondo de ojo como los nombrados anteriormente. 

Para la eliminaci\'on o extracci\'on del ruido se tuvieron en cuenta dos algoritmos: filtro de Difusi\'on Anisotr\'opica y filtro de  Coherencia.\\

En la actualidad existe una  amplia gama de metodologías para la detección automática de defectos  con  bajo contraste [1s-5], todas ellas dependen de la buena adquisición de la imagen, las condiciones de iluminación, el tipo de material que se desea inspeccionar, entre otros. A esto se le suma las características de los defectos que se quieran detectar, un ejemplo de estas características es el contraste que generan los defectos con respecto al fondo de la muestra, defectos que generan un mayor contraste son más fáciles de detectar que aquellos cuyo 
contraste es muy bajo.\\
Los defectos de bajo contraste se pueden clasificar dentro de dos clases con características diferentes para las etapas posteriores del proceso como la segmentación y clasificación. En la primera clase se encuentran los defectos oscuros de bajo contraste, estos se caracterizan por tener un nivel de gris más bajo. La segunda clase está conformada por defectos claros con bajo contraste, los cuales tienen un nivel de gris más alto 
en comparación al fondo en el que se encuentran embebidos. Se debe tener en cuenta que los defectos tienen niveles de gris diferentes (mayor o menor) al del resto del objeto de estudio, y  que el bajo contraste del defecto con respecto al  fondo del objeto  hace que la tarea de detección sea compleja tanto para las personas como para los sistemas de inspección industrial basados en visión por computador.\\

Esta secci\'on se centra en la evaluaci\'on del filtro de difusión anisotrópico como estrategia de realzado previa a la segmentación de defectos oscuros de bajo contraste en objetos pequeños de alta reflectividad con iluminación no homogénea.

Definici\'on del modelo de difusi\'on anisotr\'opico\\

La difusión anisotrópica propuesta por Perona y Malik \cite{perona1990scale} está dada de la siguiente manera: 

\begin{displaymath}
I_t=div\lbrack c_t(x,y)\nabla I_t(x,y)\rbrack  \hspace{2cm}(1)
\end{displaymath}

Donde div  es el operador de divergencia, \begin{displaymath}  \Delta \end{displaymath} es el operador Laplaciano y \begin{displaymath}
\nabla \end{displaymath}   es el operador gradiente.  \begin{displaymath} c_t(x,y)\end{displaymath} es el coeficiente 
de difusión definido en función del gradiente de tal forma que se adapte para que los bordes entre regiones sean preservados 
y los detalles intraregiones sean suavizados, \begin{displaymath} c_t(x,y)\end{displaymath} se define como se muestra a continuación:

\begin{displaymath}
c_t(x,y)=g(\nabla I_t(x,y))=1/\lbrack1+(\left|\nabla I_t/k\right|)^2\rbrack \hspace{2cm}(2)
\end{displaymath}

La difusión anisotrópica  se puede expresar de forma discreta de la siguiente manera:

\begin{displaymath}
I_{t+1}(x,y)=I_t(x,y) + \frac14{{{\sum_{i=1}^{4} \lbrack c_t^i(x,y).\nabla I_t^i(x,y)\rbrack}}} \hspace{2cm}(3)
\end{displaymath}

En el modelo discreto \begin{displaymath} \nabla I_t^i(x,y), i= 1,2,3,...,4 \end{displaymath} son los gradientes de los vecindarios en diferentes direcciones y \begin{displaymath} c_t^i(x,y) \end{displaymath} es el coeficiente de difusi\'on mencionado en (2), en esta ecuaci\'on k es una constante que se debe fijar de acuerdo a la aplicaci\'on del filtro seg\'un el desempe\~no que se busque.\\

El filtro de Difusi\'on anisotr\'opica es una técnica destinada a reducir el ruido de una imagen sin necesidad de retirar las piezas importantes del contenido de la imagen, por lo general los bordes, líneas u otros detalles que son importantes para la interpretación de la imagen, el modelo de difusión anisotrópico asemeja el proceso que crea un espacio de escala, donde una imagen genera una familia parametrizada de forma sucesiva cada vez más borrosas de imágenes, basadas en un proceso de difusión.\\ 

EXPLICAR UN POCO MAS DIFUSION ANISOTROPICA

El filtro de Coherencia ...This function COHERENCEFILTER will perform Anisotropic Diffusion of a 2D gray/color image or 3D image volume, Which will reduce the noise in an image while preserving the region edges, and will smooth along
the image edges removing gaps due to noise.\\ HACERLO COMPLETO 



Para la eliminaci\'on o extracci\'on del fondo se tuvieron en cuenta tres filtros denominados de paso bajo, el filtro de mediana, el filtro de media y el filtro gaussiano. 

El proceso de filtrado consiste en la aplicación a cada uno de los pixels de la imagen de una matriz de filtrado de tamaño N x N (generalmente de 3x3 aunque puede ser mayor) compuesta por números enteros y que genera un nuevo valor mediante una función del valor original y los de los pixels circundantes. El resultado final se divide entre un escalar, generalmente la suma de los coeficientes de ponderación. Los filtros se pueden expresar mediante una ecuación (6.1)

\begin{displaymath}
ND'_{i,j}=\frac{ND_{i-1,j-1}+ND_{i,\;j-1}+ND_{i+1,j-1}+ND_{i-1,j}+ND_{i,j}+ND_{i+1,j}+ND_{i-1,j+1}+ND_{i-1,j+1}ND_{i-1,j+1}}9 \hspace{2cm}(6.1)
\end{displaymath}

donde i y j representan la fila y la columna de cada pixel,  \[ND_{i;j}\] su Nivel Digital y \[ND'_{i,j}\] el Nivel Digital obtenido tras hacer el filtrado.\\

El objetivo de estos consiste en suavizar la im\'agen, son útiles cuando se supone que la imagen tiene gran cantidad de ruido y se quiere eliminar. También pueden utilizarse para resaltar la información correspondiente a una determinada escala (tamaño de la matriz de filtrado); 


\begin{itemize}
	\item[$*$] Filtro de la media, asigna al pixel central la media de todos los pixeles incluidos en la ventana. La matriz de filtrado estaría compuesta por unos y el divisor sería el número total de elementos en la matriz.
	
\begin{displaymath}
\begin{array}{l}\begin{array}{cccc}20&23&30&31\\22&21&29&30\\23&24&32&33\\29&31&34&37\end{array}\rightarrow\begin{array}{ccc}1&1&1\\1&1&1\\1&1&1\end{array}\rightarrow\begin{array}{cccc}N&N&N&N\\N&24.8&28.1&N\\N&27.2&30.1&N\\N&N&N&N\end{array}\\\\\;\;\;\;\;\;\;\;\;\;\;\;\;\;\;\;\;\;\;\;\;\;\;\;\;\;\;\;\text{Filtro de Media}\end{array}
\end{displaymath}
				
	\item[$*$] Filtro de la mediana tiene la ventaja de que el valor final del pixel es un valor real presente en la imagen y no un promedio, de este modo se reduce el efecto borroso que tienen las imagenes que han sufrido un filtro de media. Además el filtro de la mediana es menos sensible a valores exremos. El incoveniente es que resulta más complejo de calcular ya que hay que ordenar los diferentes valores que aparecen en los pixeles incluidos en la ventana y determinar cual es el valor central.
\begin{displaymath}
\begin{array}{l}\begin{array}{cccc}20&23&30&31\\22&21&29&30\\23&24&32&33\\29&31&34&37\end{array}\rightarrow\begin{array}{cccc}N&N&N&N\\N&23&30&N\\N&29&31&N\\N&N&N&N\end{array}\\\\\;\;\;\;\;\;\;\;\;\;\;\;\;\;\;\;\;\;\text{Filtro de Mediana}\end{array}
\end{displaymath}

	\item[$*$]Filtros gaussianos. Simulan una distribución gaussiana bivariante. El valor máximo aparece en el pixel central y disminuye hacia los extremos tanto más rápido cuanto menor sea el parámetro de desviación típica s. El resultado será un conjunto de valores entre 0 y 1. Para transformar la matriz a una matriz de números enteros se divide toda la matriz por el menor de los valores obtenidos. La ecuación para calcularla es:
	
	\begin{displaymath}
	g(x,y)=e^{-\frac{x^2+y^2}{2\ast s^2}} \hspace{2cm}(6.2)
	\end{displaymath}
	
	\begin{displaymath}
		G(x,y)=\frac{g(x,y)}{min_{x,y}(g(x,y))} \hspace{2cm}(6.3)
	\end{displaymath}
\end{itemize}


Para mejorar el contraste de las im\'agenes se utilizo la t\'ecnica Contrast Limited AHE (CLAHE) -  differs from ordinary adaptive histogram equalization in its contrast limiting. This feature can also be applied to global histogram equalization, giving rise to contrast limited histogram equalization (CLHE), which is rarely used in practice. In the case of CLAHE, the contrast limiting procedure has to be applied for each neighbourhood from which a transformation function is derived. CLAHE was developed[3] to prevent the overamplification of noise that adaptive histogram equalization can give rise to.

This is achieved by limiting the contrast enhancement of AHE. The contrast amplification in the vicinity of a given pixel value is given by the slope of the transformation function. This is proportional to the slope of the neighbourhood cumulative distribution function (CDF) and therefore to the value of the histogram at that pixel value. CLAHE limits the amplification by clipping the histogram at a predefined value before computing the CDF. This limits the slope of the CDF and therefore of the transformation function. The value at which the histogram is clipped, the so-called clip limit, depends on the normalization of the histogram and thereby on the size of the neighbourhood region. Common values limit the resulting amplification to between 3 and 4.

It is advantageous not to discard the part of the histogram that exceeds the clip limit but to redistribute it equally among all histogram bins.

The redistribution will push some bins over the clip limit again (region shaded green in the figure), resulting in an effective clip limit that is larger than the prescribed limit and the exact value of which depends on the image. If this is undesirable, the redistribution procedure can be repeated recursively until the excess is negligible.

Con los métodos de filtrado definidos y analizados teniendo en cuenta sus ventajas y desventajas, se realizaron cuatro pipelines para determinar cual de estos es el mejor para utilizar en los sets de im\'agenes propuestos.

\begin{itemize}
	    \item Pipeline 1 -> Extracci\'on de fondo  +  Realce  +  Extracci\'on de ruido
		\item Pipeline 2 -> Realce +  Extracci\'on de ruido
		\item Pipeline 3 -> Realce +  Extracci\'on de fondo
		\item Pipeline 4 -> Realce  + Extracci\'on de fondo   +  Extracci\'on de ruido
\end{itemize}


Inicialmente se obtiene el canal verde de la im\'agen, el cual permite observar de manera mas clara los vasos sangu\'ineos de la misma. Una vez realizdo esto, se procesan las im\'agenes por los pipelines nombrados anteriormente.
A continuaci\'on se describe cada etapa, detallando de que se encarga cada una de ellas:

\begin{itemize}
	\item Extracci\'on de fondo: para esta etapa se realizaron pruebas con los tres filtros nombrados mas arriba. El objetivo de  esta etapa consiste en suavizar la im\'agen, para esto calculamos el fondo de la imagen, luego le restamos a la imagen original este fondo y de esta manera podemos remover el efecto de bias causado por la modalidad de captura de la imagen. Finalmente se convierte la imagen a double para tener mayor precisión en las próximas etapas.\\
Codigo de media, mediana y gaussiano\\
	\item Extracción de ruido:  Esta etapa consiste en aplicar un filtro que sea capaz de eliminar el ruido provocado por el método de captura o procesamiento de la  imagen, como así también el ruido ocasionado por la suciedad del lente, partículas en suspensión, etc. 
Para esta etapa se aplicaron los filtros de difusión anisotropica y filtro anisotrópico de realce de coherencia con el esquema de difusión isotrópica de Perona y Malik.
	\item Realce: Esta función realiza la Ecualización Adaptativa de la imagen permitiendo percibir detalles cuando el fondo no es homogéneo. Es decir, que el realce pretende mejorar el contraste de una imagen, haciendo que "los píxeles claros se aclaren más" y "los píxeles oscuros se oscurezcan". 
\end{itemize}

Para determinar cual de los pipeline es el que permite obtener el mejor preprocesado de las imágenes, se fueron variando los parametros correspondientes a cada filtro y analizando el resultado del mismo, teniendo en cuenta el valor de área promedio bajo la curva obtenido.

El resultado se logra comparando la imágen preprocesada con la imágen del ground true correspondinte. Esta comparación se realza con el método vlroc provisto por Matlab,  herramienta de software matemático que ofrece un entorno de desarrollo integrado (IDE) con un lenguaje de programación propio (lenguaje M). 

El área bajo la curva, también llamada región de interés, devuelve un valor acotado entre 0 y 1, siendo 1 el mejor y mayor valor alcanzable. Este valor de área es un resultado promedio, ya que se procesan todas las imágenes pertenecientes a un mismo dataset y finalmente se obtiene este valor que luego es utilizado para el análisis de la imágen.\\

Para cada conjunto de imágenes se aplicaron los pipelines descritos anteriormente.\\
En cuanto a la variación de parámetros en la extracción de fondo, se realizó una busqueda exhaustiva para obtener la mejor ventana para cada uno de los sets. Para esto se realizó una iteración sobre el total de imágenes para distintos tamaños de ventanas y finalmente se encontró el óptimo para cada uno. Por razones de costo computacional, las iteraciones para la obtención de la mejor ventana, se realizó con tamaños de ventanas en el rango de 3 a 400. Inicialmente se iteró en este rango con saltos de 20 pasos, para así poder encontrar la curva aproximada con el valor máximo y luego disminuir el rango como así también los saltos.\\

Para la extracción de ruido se utilizó un procedimiento similar para obtener el mejor numero de iteraciones para un mejor resultado. A diferencia del procedimiento de extracción de fondo, este es un proceso computacionalmente costoso, encontrandose el filtro de Coherencia muy por encima del filtro de difusión anisotrócipa.
En el filtro de coherencia, se realizaron iteraciones de hasta 105, pero luego se disminuyo a 50 iteraciones debido a que en relación costo-beneficio no hay mejoras sustanciales para numeros altos de iteraciones.

Tambien se realizaron pruebas, para determinar el valor del parámetro kappa, se observo que este parámetro entrega su mejor resultado cuando es igual a dos.


\subsection{Dataset ARIA}

A continuación se muestran resultados logrados en el preprocesado de las imágenes para la obtención del tamaño de ventana óptimo para el dataset Aria.\\

Extracción de fondo
\begin{itemize}
	\item[$*$]Mediana 
\end{itemize}
Se realiz\'o una iteraci\'on con el tamaño de ventana variando desde 3 hasta 200, con pasos de a 10. De esta forma se obtuvo el siguiente resultado(\ref{fig:MedianaRangoGrande})\\

\begin{figure}[H]
	{
	\centering
	\includegraphics[width=1\textwidth]{Figures/MedianaRangoGrande}
	\caption[Ventana Mediana 0-200]{\'Areas promedio para diferentes tamaños de ventana, en el rango 0-200.}
	\label{fig:MedianaRangoGrande}
	}
\end{figure}	

De la im\'agen anterior se puede ver que el valor m\'aximo se encuentre en un rango mas chico, por esta  raz\'on se realiza otra iteraci\'on con un tamaño de ventanas variando de 15 a 60. De este modo, mirando el gr\'afico obtenido (\ref{fig:MedianaRangoChico}) se puede observar que  el mejor tama\~no de ventana es 41 ya que es el punto m\'aximo de la par\'abola con un valor de \'area bajo la curva de 0.8292.

\begin{figure}[H]
	{
	\centering
	\includegraphics[width=1\textwidth]{Figures/MedianaRangoChico}
	\caption[Ventana Mediana 15-60]{\'Areas promedio para diferentes tamaños de ventana, en el rango 15-60.}
	\label{fig:MedianaRangoChico}
	}
\end{figure}	

\begin{itemize}
	\item[$*$]Media
\end{itemize}

Para la b\'usqueda de la ventana \'optima en la media, se realiz\'o en primer lugar una iteraci\'on en un rango desde 7 hasta 550 con pasos de a 20 obteni\'endose el siguiente resultado (\ref{fig:MediaRangoGrande}):

\begin{figure}[H]
	{
	\centering
	\includegraphics[width=1\textwidth]{Figures/MediaRangoGrande}
	\caption[Ventana Media 7-550]{\'Areas promedio para diferentes tama\~nos de ventana, en el rango 7-550.}
	\label{fig:MediaRangoGrande}
	}
\end{figure}

De la im\'agen anterior se puede ver que el valor m\'aximo se encuentra en un rango mas chico, por esta  raz\'on se realiza otra iteraci\'on con un tama\~no de ventana variando de 15 a 80. De este modo, mirando el gr\'afico obtenido  (\ref{fig:MediaRangoChico}) se puede observar que  el mejor tama\~no de ventana es 41 ya que es el punto m\'aximo de la par\'abola con un valor de \'area bajo la curva de 0.7435.

\begin{figure}[H]
	{
	\centering
	\includegraphics[width=1\textwidth]{Figures/MediaRangoChico}
	\caption[Ventana Media 15-80]{\'Areas promedio para diferentes tama\~nos de ventana, en el rango 15-80.}
	\label{fig:MediaRangoChico}
	}
\end{figure}

\begin{itemize}
	\item[$*$]Gaussiano
\end{itemize}

Para el filtro gaussiano no hay parámetros que varíen significativamente el resultado, por lo que solo se evaluó el resultado de aplicar el filtro con el mismo proceso mencionado para la media y mediana. 

A continuaci\'on se puede observar (\ref{fig:gaussiano}) que para las distintas imágenes no existe variación del resultado, y el área obtenida es 0.5469.


\begin{figure}[H]
	{
	\centering
	\includegraphics[width=1\textwidth]{Figures/Gaussiano}
	\caption[Gaussiano]{Areas promedio para cada imagen aplicando el filtro Gaussiano.}
	\label{fig:gaussiano}
	}
\end{figure}

Extracci\'on de ruido\\

En esta etapa se analiz\'o el mejor numero de iteraci\'on para los filtros de difusi\'on anisotr\'opica y filtro de coherencia en base a los resultados obtenidos. Para obtener un resultado \'optimo se realizaron iteraciones varias hasta encontrar el punto de inflexi\'on para este conjunto de im\'agenes.\\


\begin{itemize}
	\item[]Filtro de Difusi\'on Anisotr\'opica
\end{itemize}

Para las pruebas realizadas con este filtro se tuvo en cuenta la variaci\'on del parametro que hace referencia al n\'umero de iteraciones. Las mismas se variaron en un rango de entre 1 y 210 con pasos de 40,obteniendose de este modo, la gr\'afica siguinte (\ref{fig:anisotropica}):

\begin{figure}[H]
	{
	\centering
	\includegraphics[width=1\textwidth]{Figures/AnisotropicoRangoGrande}
	\caption[Difusión Anisotrópica]{Areas promedio para distintos valores de iteracion, en un rango 0-210.}
	\label{fig:anisotropica}
	}
\end{figure}

Con lo que se determino que los mejores resultados se obtienen entre los valores 1 y 50. Al repetir el proceso aplicando el filtro con cada valor de iteración entre el rango 1-50 se obtuvo:

\begin{figure}[H]
	{
	\centering
	\includegraphics[width=1\textwidth]{Figures/AnisotropicoRangoChico}
	\caption[Area promedio en el rango 1 -50]{Areas promedio para distintos valores de iteracion, en un rango 1-50.}
	}
\end{figure}

Como podemos observar, el mejor resultado se obtuvo con la iteración 1 y un valor de  área promedio de 0.7557.

\begin{itemize}
	\item[]Filtro de Coherencia
\end{itemize}

El proceso para el filtro de coherencia se reali\'o de la misma forma que el de difusi\'on anisotr\'opica con iteraciones entre 3 y 450 con pasos de 50. Los resultados obtenidos fueron los siguientes (\ref{fig:coherencia}):

\begin{figure}[H]
	{
	\centering
	\includegraphics[width=1\textwidth]{Figures/CoherenciaRangoGrande}
	\caption[\'Area promedio del Filtro Cohernecia en el rango 0-550]{Areas promedio para distintos valores de iteracion, en un rango 0-550.}
	\label{fig:coherencia}
	}
\end{figure}

Observando el resultado se conluy\'o que los mejores resultados se obtienen con valores que iteran entre 140 y 160, por lo que se itero por cada uno de estos valores y se obtuvo (\ref{fig:coherenciaRangoChico}):

\begin{figure}[H]
	{
	\centering
	\includegraphics[width=1\textwidth]{Figures/CoherenciaRangoChico}
	\caption[\'Area promedio del filtro de coherencia en el rango 140-160]{\'Areas promedio para distintos valores de iteracion, en un rango 140-160.}
	\label{fig:coherenciaRangoChico}
	}
\end{figure}

Finalmente se determin\'o que para el filtro de coherencia el parámetro de iteraciones consigue su valor \'optimo en 154 iteraciones, obteniendo un área de 0.8756 para este conjunto de im\'agenes.\\

A partir de este  \'analisis realizado en cada una de las etapas, se realizaron las pruebas de los pipeline definidos al principio. 


Como conclusi\'on podemos ver que en la fase de extracci\'on de fondo los mejores resultados se obtuvieron con la mediana, por tal raz\'on se utiliz\'o la misma para el objetivo de esta etapa con una ventana de 41x41. Para la fase de extracci\'on de ruido se observa que los mejores resultados se obtienen con el filtro de coherencia con 154 iteraciones. La etapa de realce simplemente utiliza la funci\'on adapthisteq que provee Matlab.\\

Teniendo los par\'ametros  anteriores, los resultados de los diferentes pipeline, fueron los siguientes:\\


\begin{table}[!ht]
  \begin{center}
    \begin{tabular}{| l c |}
    \hline
    Pipeline &   Area \\
    Pipeline 1 & 0.86\\
    Pipeline 2 & 0.48\\
    Pipeline 3 & 0.83\\
    Pipeline 4 & 0,87\\
    \hline
    \end{tabular}
  \end{center}
  \caption{Dataset ARIA}
\end{table}



\subsection{Dataset HRF}

Para este conjunto de im\'agenes se utilizaron 3 tipos de im\'agenes, de las cuales 5 se corresponden con la enfermedad glaucoma, 5 con rentinopa\'ia diabetica y las restantes 5 son im\'agenes sanas. 

El proceso de las pruebas fue el mismo que se utiliz\'o para el dataset aria. Partiendo de una im\'agen con el canal verde, se realizaron las pruebas correspondientes a cada uno de los pipeline mencionados anteriormente, obteniendose los siguientes resultados.\\

Extracción de fondo
\begin{itemize}
	\item[$*$]Mediana 
\end{itemize}

En este caso para la obtenci\'on de la mejor ventana a utilizar en este dataset, se realiz\'o la b\'usqueda de la misma haciendo iteraciones que fueron desde 3 hasta 403 con pasos de 20. Finalizado esta prueba se obtuvo el siguiente resultado (\ref{fig:MedianaRangoGrandeHRF}): 

\begin{figure}[H]
	{
	\centering
	\includegraphics[width=1\textwidth]{Figures/MedianaRangoGrandeHRF}
	\caption[Mediana HRF]{\'Areas promedio para distintos valores de iteracion, en un rango 3 - 403.}
	\label{fig:MedianaRangoGrandeHRF}
	}
\end{figure}


De la im\'agen anterior se puede ver que el punto m\'aximo de la par\'abola se encuentra en un rango menor, por tal raz\'on se realiza otra iteraci\'on en el rango 79-141 con pasos de a 2. La im\'agen a continuaci\'on (\ref{fig:MedianaRangoChicoHRF}) nos muestra el resultado de la mejor ventana para la mediana, siendo esta 103.\\


\begin{figure}[H]
	{
	\centering
	\includegraphics[width=1\textwidth]{Figures/MedianaRangoChicoHRF}
	\caption[Mediana HRF]{\'Areas promedio para distintos valores de iteracion, en un rango 79 - 141.}
	\label{fig:MedianaRangoChicoHRF}
	}
\end{figure}



\begin{itemize}
	\item[$*$]Media
\end{itemize}

Para la b\'usqueda de la ventana \'optima en el caso de la media, se realiz\'o una iteraci\'on inicial que se encuentra en los valores de 3 a 213 con pasos de 20, observandose (\ref{fig:MediaRangoGrandeHRF}) que el valor \'optimo puede encontrarse en un rango mas pequeño, se realiz\'o la siguiente iteraci\'on con valores de entre 35-149 con pasos de a 2 (\ref{fig:MediaRangoChicoHRF}). De esta forma se encontr\'o la mejor ventana, teniendo la misma un valor de 83.

\begin{figure}[H]
	{
	\centering
%	\begin{subfigure}[b]{1\textwidth}
	\includegraphics[width=1\textwidth]{Figures/MediaRangoGrandeHRF}
	\caption[Media HRF]{\'Areas promedio para distintos valores de iteracion, en un rango 3 - 213.}
	\label{fig:MediaRangoGrandeHRF}
%	\end{subfigure}
	
%	\begin{subfigure}[b]{0.3\textwidth}
	\includegraphics[width=1\textwidth]{Figures/MediaRangoChicoHRF}
	\caption[Media HRF]{\'Areas promedio para distintos valores de iteracion, en un rango 35 - 149.}
	\label{fig:MediaRangoChicoHRF}
%	\end{subfigure}
	}
\end{figure}

\begin{itemize}
	\item[$*$]Gaussiano
\end{itemize}

En el filtro gaussiano no hay parámetros que varíen significativamente el resultado, por lo que solo se evaluó el resultado de aplicar el filtro con el mismo proceso mencionado para la media y mediana.\\


Como conclusi\'on podemos ver que en la fase de extracci\'on de fondo los mejores resultados se obtuvieron con la mediana, por tal raz\'on se utiliz\'o la misma para el objetivo de esta etapa con una ventana de 103x103. Para la fase de extracci\'on de ruido se observa que los mejores resultados se obtienen con el filtro de coherencia con 50 iteraciones. La etapa de realce simplemente utiliza la funci\'on adapthisteq que provee Matlab.\\

Teniendo los par\'ametros  anteriores, los resultados de los diferentes pipeline, fueron los siguientes:\\


\begin{table}[!ht]
  \begin{center}
    \begin{tabular}{| l c |}
    \hline
    Pipeline &   Area \\
    Pipeline 1 & 0.94650128394401889\\
    Pipeline 2 & 0.30937490250808752\\
    Pipeline 3 & 0.93866336888430779\\
    Pipeline 4 & 0,9539922\\
    \hline
    \end{tabular}
  \end{center}
  \caption{Dataset HRF}
\end{table}



\begin{itemize}
	\item Pipeline 1
		\begin{itemize}
			\item	Extracci\'on de fondo (mediana) + Realce + Sacar Ruido (Anisotrópica): en este caso seutiliz\'o la la extracc se utilizó para 	sacar el ruido, la función anisotrópica con un valor de kappa=2 y número de iteraciones igual a 20, se obtuvo así un área promedio de 0.88634779241998385 para la iteración 1. Para valores mayores de kappa, disminuye el área promedio.
			\item	Sacar fondo (mediana) + Realce + Sacar Ruido (Coherencia): en este caso se realizaron 50 iteraciones, ya que en pruebas independientes se vio que a mayor número de iteraciones, mejores eran los resultados. El área promedio obtenida fue de 0.94650128394401889 para la iteración 50. Si bien el resultado aumenta hasta la iteración 105, no vale la pena llegar a tal punto ya que el aumento no es significativo con respecto al tiempo que tarda el algoritmo en realizar dichas iteraciones.
			\item Sacar fondo (media) + Realce + Sacar Ruido (Anisotrópica): en este caso se utilizó un kappa con valor igual a 2 y un número de iteraciones igual a 20. El área promedio obtenida fue de 	 	0.86863159426770764 para 20 iteraciones.
			\item Sacar fondo (media) + Realce + Sacar Ruido (Coherencia): se realizaron 50 iteraciones y se obtuvo un área promedio de 0.92994809994049088.
	\end{itemize}
	\item Pipeline 2
		\begin{itemize}
			\item Realce + Sacar Ruido (Anisotrópica): la función anisotrópica se realizó con 20 iteraciones para un valor de kappa igual a 2. El valor de área promedio obtenido fue de 0.3097178264596 para todas las iteraciones.
			\item	Realce + Sacar Ruido (Coherencia): la prueba se realizó con 50 iteraciones y se obtuvo un área promedio de  0.30937490250808752.
		\end{itemize}
	\item Pipeline 3
		\begin{itemize}
			\item Realce + Sacar fondo (mediana): En este caso se obtuvo un área promedio de 0.93866336888430779.
			\item Realce + Sacar fondo (media): en este caso se obtuvo un área promedio de 0.92444315570164037.
			\item Realce + Sacar fondo (Gaussiano): en este caso se obtuvo un área promedio de 0.48294301483035912. Como el valor obtenido es muy bajo respecto a los otros métodos utilizados para sacar el fondo, este método no se calculo más.
		\end{itemize}
	\item Pipeline 4
		\begin{itemize}
			\item Realce + Sacar fondo (media) + sacar ruido (anisotrópica): en este caso se obtuvo un área promedio de 0.89198264475071964, haciendo 20 iteraciones y con el valor de kappa igual a dos.
			\item Realce + Sacar fondo (mediana) + sacar ruido (anisotrópica): en este caso se obtuvo un área promedio de 0.896901101865378 con 20 iteraciones y el valor de kappa igual a dos.
			\item Realce + Sacar fondo (mediana) + sacar ruido (coherencia): en este caso, la prueba se realizó con 50 iteraciones y se obtuvo un área promedio de 0,9539922.
			\item Realce + Sacar fondo (media) + sacar ruido (coherencia): olvide hacer esta prueba, pero se ve en todos los casos que para sacar fondo, la mejor opción es la mediana.
		\end{itemize}
\end{itemize}


\subsection{Dataset DRIVE}


%-----------------------------------
%	SECTION 3
%-----------------------------------

\section{Extracci\'on de caracter\'isticas}


%----------------------------------------------------------------------------------------
%	SECTION 4
%----------------------------------------------------------------------------------------

\section{M\'etodo de segmentaci\'on}

