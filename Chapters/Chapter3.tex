% Chapter Template

\chapter{M\'etodos} % Main chapter title

\label{Chapter3} % Change X to a consecutive number; for referencing this chapter elsewhere, use \ref{ChapterX}

%----------------------------------------------------------------------------------------
%	SECTION 1
%----------------------------------------------------------------------------------------

\section{Descripci\'on general}

En el desarrollo inicial de este trabajo final se utilizaron diferentes m\'etodos para determinar el mejor preprocesamiento a utilizar en los sets de im\'agenes propuestos, HRF, ARIA y DRIVE. Los mismos contienen im\'agenes con diferentes enfermedades, tales como, retinopat\'ia diab\'etica, edema macular asociado a la edad e im\'agenes sanas.
Los m\'etodos utilizados en el preprocesamiento de las im\'agenes fueron varios.

%-----------------------------------
%	SECTION 2
%-----------------------------------
\section{Preprocesamiento}
En primer lugar y teniendo en cuenta los m\'etodos a utilizar para cada etapa del preprocesamiento, se determinaron los m\'etodos a utilizar para cada una de estas. Si dividimos el preprocesamiento en etapas, se destacan 3, sacar fondo, sacar ruido, realce de bordes. Para la etapa de extracci\'on de fondo se utilizaron 3 m\'etodos: mediana, media y gaussiano. Para la etapa de extracci\'on de ruido se utilizaron 2 m\'etodos: Anisotr\'opico y Coherencia. Por \'ultimo para el realce de de bordes se aplico el m\'etodo Clahe.

Con estas etapas definidas, se realizaron diferentes pipeline a analizar para determinar el mejor pipeline para cada uno de los sets de im\'agenes nombrados anteriormente.



%-----------------------------------
%	SECTION 3
%-----------------------------------

\section{Extracci\'on de caracter\'isticas}


%----------------------------------------------------------------------------------------
%	SECTION 4
%----------------------------------------------------------------------------------------

\section{M\'etodo de segmentaci\'on}

