% Chapter Template

\chapter{M\'etodos} % Main chapter title

\label{Chapter3} % Change X to a consecutive number; for referencing this chapter elsewhere, use \ref{ChapterX}

%----------------------------------------------------------------------------------------
%	SECTION 1
%----------------------------------------------------------------------------------------

\section{Descripci\'on general}

Para la obtenci\'on de una im\'agen que es utilizada por personal de la salud para la detecci\'on de una enfermedad o la realizaci\'on de un diagnostico prec\'oz respecto a un afecci\'on que sufre el paciente, es necesario realizar un preprocesamiento de las imagenes de fondo de ojo teniendo en cuenta que la misma es de suma importancia para la atenci\'on y posible tratamiento de una persona con problemas visuales severos o no.

Con el fin de generar una im\'agen que permita al Medico proveer ayuda en su diagn\'ostico, es necesario preprocesar la captura de fondo de ojo y de esta forma acercarle una im\'agen mas clara que la obtenida por la herramienta utilizada en la captura de la misma.




%-----------------------------------
%	SECTION 2
%-----------------------------------
\section{Preprocesamiento}

El preprocesamiento digital de im\'agenes permite mejorar distintos aspectos de la misma, de este modo se puede determinar con mayor efectividad el diagn\'ostico correspondiente.

En este sentido, esta etapa de preprocesamiento busca mejorar algunos de los siguientes:

\begin{itemize}
	\item[$*$] Suavizar la im\'agen:  reducir la cantidad de variaciones de intensidad entre píxeles vecinos.					\item[$*$] Eliminar ruido: eliminar aquellos píxeles cuyo nivel de intensidad es muy diferente al de sus vecinos y cuyo origen puede estar tanto en el proceso de adquisición de la imagen como en el de transmisión.
	\item[$*$] Realzar bordes: destacar los bordes que se localizan en una imagen.
	\item[$*$] Detectar bordes: detectar los píxeles donde se produce un cambio brusco en la función intensidad.
\end{itemize}

En el desarrollo inicial de este trabajo se utilizaron diferentes m\'etodos para determinar el mejor preprocesamiento a utilizar en los sets de im\'agenes propuestos, HRF, ARIA y DRIVE. Los mismos contienen im\'agenes con diferentes enfermedades, tales como, retinopat\'ia diab\'etica, edema macular asociado a la edad e im\'agenes sanas entre otras. 

En primer lugar, se determinaron los algoritmos a utilizar para mejorar el preprocesamiento de las im\'agenes, es decir, disminuir los efectos secundarios que se generan en la captura de im\'agenes de fondo de ojo como los nombrados anteriormente. 

Para la eliminaci\'on o extracci\'on del ruido se tuvieron en cuenta dos algoritmos, filtro de Difusi\'on Anisotr\'opica y filtro de  Coherencia.\\

El filtro de Difusi\'on anisotr\'opica es una técnica destinada a reducir el ruido de una imagen sin necesidad de retirar las piezas importantes del contenido de la imagen, por lo general los bordes, líneas u otros detalles que son importantes para la interpretación de la imagen 1 2 el modelo de difusión anisotrópico asemeja el proceso que crea un espacio de escala, donde una imagen genera una familia parametrizada de forma sucesiva cada vez más borrosas de imágenes, basadas en un proceso de difusión(Ref WIKIPEDIA).\\

El filtro de Coherencia ...This function COHERENCEFILTER will perform Anisotropic Diffusion of a 2D gray/color image or 3D image volume, Which will reduce the noise in an image while preserving the region edges, and will smooth along
the image edges removing gaps due to noise.\\

Para la eliminaci\'on o extracci\'on del fondo se tuvieron en cuenta tres filtros denominados de paso bajo, el filtro de mediana, el filtro de media y el filtro gaussiano. 

El proceso de filtrado consiste en la aplicación a cada uno de los pixels de la imagen de una matriz de filtrado de tamaño N x N (generalmente de 3x3 aunque puede ser mayor) compuesta por números enteros y que genera un nuevo valor mediante una función del valor original y los de los pixels circundantes. El resultado final se divide entre un escalar, generalmente la suma de los coeficientes de ponderación. Los filtros se pueden expresar mediante una ecuación (6.1)

\begin{displaymath}
N D'_i,_j=\frac{ND_{i-1},_{j-1} + ND_{i},_{j-1} + ND_{i+1},_{j-1} + ND_{i-1},_{j} + ND_{i},_{j} + ND_{i+1},_{j} + ND_{i-1},_{j+1} + ND_{i-1},_{j+1} + ND_{i-1},_{j+1}}{9}
\end{displaymath}

donde i y j representan la fila y la columna de cada pixel, N \[D_{i;j}\] su Nivel Digital y ND 0 i;j el Nivel Digital obtenido tras hacer el filtrado.\\

El objetivo de estos consiste en suavizar la im\'agen, son útiles cuando se supone que la imagen tiene gran cantidad de ruido y se quiere eliminar. También pueden utilizarse para resaltar la información correspondiente a una determinada escala (tamaño de la matriz de filtrado); 

\begin{itemize}
	\item[$*$] Filtro de la media, asigna al pixel central la media de todos los pixeles incluidos en la ventana. La matriz de filtrado estaría compuesta por unos y el divisor sería el número total de elementos en la matriz.					\item[$*$] Filtro de la mediana tiene la ventaja de que el valor final del pixel es un valor real presente en la imagen y no un promedio, de este modo se reduce el efecto borroso que tienen las imagenes que han sufrido un filtro de media. Además el filtro de la mediana es menos sensible a valores exremos. El incoveniente es que resulta más complejo de calcular ya que hay que ordenar los diferentes valores que aparecen en los pixeles incluidos en la ventana y determinar cual es el valor central.
	\item[$*$]Filtros gaussianos. Simulan una distribución gaussiana bivariante. El valor máximo aparece en el pixel central y disminuye hacia los extremos tanto más rápido cuanto menor sea el parámetro de desviación típica s. El resultado será un conjunto de valores entre 0 y 1. Para transformar la matriz a una matriz de números enteros se divide toda la matriz por el menor de los valores obtenidos. La ecuación para calcularla es:
\end{itemize}

(REF pdf Tecnicas de Filtrado)
IMAGENES DE MATRICES PARA MEDIA, MEDIANA Y GAUSSIANO
\begin{figure}[h]
\minipage{0.32\textwidth}
	\includegraphics[width=\linewidth]{Figures/media}
	\caption[Filtro de Media]{ (Filtro de Media).}\label{fig:Filtro de Media}
\endminipage\hfill
\minipage{0.32\textwidth}
	\includegraphics[width=\linewidth]{Figures/media}
	\caption[ARIA]{ARIA (Degeneración macular).}\label{fig:ARIA}
\endminipage\hfill
\end{figure}




Para esto se preprocesaron las im\'agenes por diferentes pipeline, con el fin de encontrar el mejor de ellos para cada set de im\'agenes propuesto.
Cada uno de estos pipelines, se realiz\'o en base a la variaci\'on de las diferentes etapas de preprocesamiento pensado para cada uno de estos sets de im\'agenes. A continuaci\'on se describen las diferentes etapas pertenecientes a la definici\'on del pipeline.


\begin{itemize}
	    \item Pipeline 1 -> Extracci\'on de fondo  +  Realce  +  Extracci\'on de ruido
		\item Pipeline 2 -> Realce +  Extracci\'on de ruido
		\item Pipeline 3 -> Realce +  Extracci\'on de fondo
		\item Pipeline 4 -> Realce  + Extracci\'on de fondo   +  Extracci\'on de ruido
\end{itemize}

Las imágenes de los distintos dataset, son imágenes RGB. Inicialmente se obtiene el canal verde de la im\'agen, el cual permite observar de manera mas clara los vasos sangu\'ineos de la misma. Una vez realizdo esto, se procesan las im\'agenes por los pipelines nombrados anteriormente.
A continuaci\'on se describe cada etapa, detallando de que se encarga cada una de ellas:

\begin{itemize}
	\item Extracci\'on de fondo: aplicarle un filtro que suavice la imagen. De esta maneta calculamos el fondo de la imagen, luego le restamos a la imagen original este fondo y de esta manera podemos remover el efecto da bias causado por la modalidad de captura de la imagen. Finalmente se convierte la imagen a doublé para tener mayor precisión en las próximas etapas. 
Para esta etapa se evaluaron tres tipos de filtros y se variaron sus parámetros para encontrar el que mejor elimina el ruido. Estos son: media, mediana y gaussiano.
	\item SACAR  RUIDO:  Esta etapa consiste en aplicar un filtro que sea capaz de eliminar el ruido provocado por el método de captura o procesamiento de la  imagen, como así también ruido ocasionado por suciedad en lente, partículas en suspensión, etc. 
Para esta etapa se aplicaron los filtros de difusión anisotropica y filtro anisotrópico de realce de coherencia con el esquema de difusión isotrópica de Perona y Malik.
	\item REALCE: Esta función realiza la Ecualización Adaptativa de la imagen permitiendo percibir detalles cuando el fondo no es homogéneo. Es decir, que el realce pretende mejorar el contraste de una imagen, haciendo que "los píxeles claros se aclaren más" y "los píxeles oscuros se oscurezcan". 
\end{itemize}

En un principio, para poder determinar cual de los pipeline es el que mejor resultado obtiene, es decir, elimina mas ruido, lo que se hizo fue determinar que filtro y con que parámetros es mejor cada una de las etapas descriptas.
Lo que se hizo para determinar que filtro y con que parámetros se eliminaba mayor cantidad de ruido, fue aplicar para cada imagen el filtro variando sus parámetros y comparando el resultado con la imagen ground true del dataset. Para realizar la comparación se utiliza el método vl roc de  Matlab. Lo que permite comparar las imágenes, lo que entrega un valor que determina el parecido entre las imágenes. La función retorna un grafico, y un valor de área, que se corresponde con el área debajo de la curva.  Esta área esta acotado entre 0 y 1, siendo 1 el mayor valor alcanzable.
Durante la iteración por los distintos valores de parámetro, se acumularon los valores obtenidos y se dividieron por la cantidad de imágenes utilizadas, de esta manera se obtiene el área promedio para ese filtro con cada valor de parámetro. De esta forma al terminar el proceso de iteración, podemos determinar que parámetro nos entrego el mejor resultado en promedio. 

En primer lugar y teniendo en cuenta los m\'etodos a utilizar para cada etapa del preprocesamiento, se determinaron los m\'etodos a utilizar para cada una de estas. Si dividimos el preprocesamiento en etapas, se destacan 3, sacar fondo, sacar ruido, realce de bordes. Para la etapa de extracci\'on de fondo se utilizaron 3 m\'etodos: mediana, media y gaussiano. Para la etapa de extracci\'on de ruido se utilizaron 2 m\'etodos: Anisotr\'opico y Coherencia. Por \'ultimo para el realce de de bordes se aplico el m\'etodo Clahe.

Con estas etapas definidas, se realizaron diferentes pipeline a analizar para determinar el mejor pipeline para cada uno de los sets de im\'agenes nombrados anteriormente.


\subsection{Pruebas sobre cada Dataset}
Para realizar las pruebas se utilizaron los dataset Aria  y HRF. Para cada conjunto de imágenes se aplicaron las pruebas descriptas y se obtuvieron los resultados que se mencionan a continuación.
Tanto para el filtro de media o mediana, el único parámetro que se puede variar es la ventana. Para esto se itero sobre el tamaño de ventana, y se calculo el área promedio de aplicar el filtro con cada determinada ventana y de esta forma determinar que tamaño de ventana era la óptima. Lo que se hace es iterar en un rango de valores amplios, realizando saltos y no iterando por cada valor, para poder determinar el rango de valores donde se encuentra el valor máximo. Para luego realizar una iteración entre los valores que acotan la zona del máximo valor, de modo de hallar el mismo.  Este proceso se realiza de este modo, debido al gran costo computacional que requiere probar con todos los valores posibles. 
Este mismo proceso se realiza para determinar también el número de iteraciones optimo que debe realizar cada algoritmo para quitar ruido, es decir, cuantas iteraciones deben realizar el filtro anisotrópico y el de coherencia para obtener el mejor resultado. Tambien se realizaron pruebas, para determinar el valor del parámetro kappa, se observo que este parámetro entrega su mejor resultado cuando es igual a dos.


\subsubsection{Dataset ARIA}

Pruebas sobre etapa SACAR  FONDO 
Mediana 
 Se itero sobre el tamaño de ventana desde 3 hasta 200, saltando de 10. De esta forma se obtuvo el siguiente resultado:

\begin{figure}[h]
	{
	\centering
	\includegraphics[width=0.50\textwidth]{Figures/MedianaRangoGrande}
	\caption[An Electron]{Areas promedio para distintos tamanos de ventana, en un rango 0-200.}
	}
\end{figure}	

De lo cual, se pudo obtener que los valores mas alto se obtenían con ventanas de tamaño 15 a 60, por lo que se itera por cada una de estas y se obtiene que el mejor tamaño de ventana es 41 con una área de 0.8292.

	\begin{figure}[h]
	{
	\centering
	\includegraphics[width=0.50\textwidth]{Figures/MedianaRangoChico}
	\caption[An Electron]{Areas promedio para distintos tamanos de ventana, en un rango 15-60.}
	}
	\end{figure}	

Media 
La primer iteración se realizo entre 7 a 550 saltando de a 20. Se obtuvo el siguiente resultado:

\begin{figure}[h]
	{
	\centering
	\includegraphics[width=0.50\textwidth]{Figures/MediaRangoGrande}
	\caption[An Electron]{Areas promedio para distintos tamanos de ventana, en un rango 0-550.}
	}
\end{figure}

Donde se pudo determinar que los mejores resultados se obtenían con ventanas con tamaño entre 15 y 80, por lo que se itero por cada una y se determino que la optima es de tamaño 41, con un área de 0.7435.

\begin{figure}[h]
	{
	\centering
	\includegraphics[width=0.50\textwidth]{Figures/MediaRangoChico}
	\caption[An Electron]{Areas promedio para distintos tamanos de ventana, en un rango 15-80.}
	}
\end{figure}

Gaussiano 
Para el filtro gaussiano no hay parámetros para variar que varíen significativamente el resultado, por lo que solo se evaluó el resultado de aplicar el filtro con el mismo proceso mencionado para la media y mediana, pero sin variar parámetros. 

\begin{figure}[h]
	{
	\centering
	\includegraphics[width=0.50\textwidth]{Figures/Gaussiano}
	\caption[An Electron]{Areas promedio para cada imagen aplicando el filtro Gaussiano.}
	}
\end{figure}

Donde se observa que para las distintas imágenes no existe variación del resultado, y el área obtenida es 0.5469.


Pruebas para la etapa SACAR  RUIDO\\
Para esta etapa realizo el mismo proceso   que para el calculo de la ventana, pero en vez de variar el tamaño de ventana,  lo que se vario fue la cantidad de iteraciones que se debe realizar el filtro. 


Anisotrópico\\
Para el filtro anisotrópico se itero entre 1 y 210 saltando de a 40, y se obtuvo:

\begin{figure}[h]
	{
	\centering
	\includegraphics[width=0.50\textwidth]{Figures/AnisotropicoRangoGrande}
	\caption[An Electron]{Areas promedio para distintos valores de iteracion, en un rango 0-210.}
	}
\end{figure}

Con lo que se determino que los mejores resultados se obtienen entre los valores 1 y 50. Al repetir el proceso aplicando el filtro con cada valor de iteración entre el rango 1-50 se obtuvo:

\begin{figure}[h]
	{
	\centering
	\includegraphics[width=0.50\textwidth]{Figures/AnisotropicoRangoChico}
	\caption[An Electron]{Areas promedio para distintos valores de iteracion, en un rango 1-50.}
	}
\end{figure}

Con lo que se obtuvo que el mejor resultado obtenido fue 1 iteración con una área de 0.7557.

Coherencia\\
El proceso para el filtro de coherencia, iterando entre 3 y 450 saltando de a 50, entrego los siguientes resultados:

\begin{figure}[h]
	{
	\centering
	\includegraphics[width=0.50\textwidth]{Figures/CoherenciaRangoGrande}
	\caption[An Electron]{Areas promedio para distintos valores de iteracion, en un rango 0-550.}
	}
\end{figure}

Con esto se determino que los mejores resultados se obtienen con valores de iteración entre 140 y 160, por lo que se itero por cada uno de estos valores y se obtuvo:

\begin{figure}[h]
	{
	\centering
	\includegraphics[width=0.50\textwidth]{Figures/CoherenciaRangoChico}
	\caption[An Electron]{Areas promedio para distintos valores de iteracion, en un rango 140-160.}
	}
	\end{figure}


Donde se determino que para el filtro de coherencia el parámetro de iteraciones tenga su valor optimo en 154 iteraciones, obteniendo un área de 0.8756.
Luego lo que se realizo, es definir que filtros que se utilizarían para cada etapa  y con que parámetros, de acuerdo a los resultados obtenidos. Para la etapa SACAR FONDO se utiliza el filtro de mediana con una ventana de 41x41. Para la etapa de SACAR  RUIDO se utiliza el filtro de Coherencia con 154 iteraciones.  La etapa de realce solo de usa la función adpthisteq que provee Matlab.

Pruebas sobre los pipeline
Lo que se hiso a continuación es pasar todas las imágenes por cada pipeline y ver con cual se obtiene mejores resultados. Estas pruebas entregaron los siguientes resultados:
Pipeline 1: Áreas  y área promedio.
Pipeline 2: Áreas  y área promedio.
Pipeline 3: Áreas  y área promedio.
Pipeline 4: Áreas  y área promedio.



\subsubsection{Dataset HRF}

Se realizaron las pruebas con un conjunto de 15 imágenes del DataSet (5 con glaucoma, 5 con retinopatía y 5 sanas).
El proceso de pruebas fue el mismo que se utilizo para el dataset aria.  A continuación se describen las pruebas realizadas, en donde todos los pipeline parten del canal verde de la imagen.
\begin{itemize}
	\item Pipeline 1
		\begin{itemize}
			\item	Sacar fondo (mediana) + Realce + Sacar Ruido (Anisotrópica): en este caso se utilizó para 	sacar el ruido, la función anisotrópica con un valor de kappa=2 y número de iteraciones igual a 20, se obtuvo así un área promedio de 0.88634779241998385 para la iteración 1. Para valores mayores de kappa, disminuye el área promedio.
			\item	Sacar fondo (mediana) + Realce + Sacar Ruido (Coherencia): en este caso se realizaron 50 iteraciones, ya que en pruebas independientes se vio que a mayor número de iteraciones, mejores eran los resultados. El área promedio obtenida fue de 0.94650128394401889 para la iteración 50. Si bien el resultado aumenta hasta la iteración 105, no vale la pena llegar a tal punto ya que el aumento no es significativo con respecto al tiempo que tarda el algoritmo en realizar dichas iteraciones.
			\item Sacar fondo (media) + Realce + Sacar Ruido (Anisotrópica): en este caso se utilizó un kappa con valor igual a 2 y un número de iteraciones igual a 20. El área promedio obtenida fue de 	 	0.86863159426770764 para 20 iteraciones.
			\item Sacar fondo (media) + Realce + Sacar Ruido (Coherencia): se realizaron 50 iteraciones y se obtuvo un área promedio de 0.92994809994049088.
	\end{itemize}
	\item Pipeline 2
		\begin{itemize}
			\item Realce + Sacar Ruido (Anisotrópica): la función anisotrópica se realizó con 20 iteraciones para un valor de kappa igual a 2. El valor de área promedio obtenido fue de 0.3097178264596 para todas las iteraciones.
			\item	Realce + Sacar Ruido (Coherencia): la prueba se realizó con 50 iteraciones y se obtuvo un área promedio de  0.30937490250808752.
		\end{itemize}
	\item Pipeline 3
		\begin{itemize}
			\item Realce + Sacar fondo (mediana): En este caso se obtuvo un área promedio de 0.93866336888430779.
			\item Realce + Sacar fondo (media): en este caso se obtuvo un área promedio de 0.92444315570164037.
			\item Realce + Sacar fondo (Gaussiano): en este caso se obtuvo un área promedio de 0.48294301483035912. Como el valor obtenido es muy bajo respecto a los otros métodos utilizados para sacar el fondo, este método no se calculo más.
		\end{itemize}
	\item Pipeline 4
		\begin{itemize}
			\item Realce + Sacar fondo (media) + sacar ruido (anisotrópica): en este caso se obtuvo un área promedio de 0.89198264475071964, haciendo 20 iteraciones y con el valor de kappa igual a dos.
			\item Realce + Sacar fondo (mediana) + sacar ruido (anisotrópica): en este caso se obtuvo un área promedio de 0.896901101865378 con 20 iteraciones y el valor de kappa igual a dos.
			\item Realce + Sacar fondo (mediana) + sacar ruido (coherencia): en este caso, la prueba se realizó con 50 iteraciones y se obtuvo un área promedio de 0,9539922.
			\item Realce + Sacar fondo (media) + sacar ruido (coherencia): olvide hacer esta prueba, pero se ve en todos los casos que para sacar fondo, la mejor opción es la mediana.
		\end{itemize}
\end{itemize}


%-----------------------------------
%	SECTION 3
%-----------------------------------

\section{Extracci\'on de caracter\'isticas}


%----------------------------------------------------------------------------------------
%	SECTION 4
%----------------------------------------------------------------------------------------

\section{M\'etodo de segmentaci\'on}

